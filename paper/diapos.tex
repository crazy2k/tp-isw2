\documentclass{beamer}
\usetheme{default}
\usepackage[utf8]{inputenc}
\usepackage[spanish]{babel}
\usepackage{hyperref}
\usepackage{wrapfig}
\usetheme{Madrid}
\usecolortheme{seahorse}
\usefonttheme{professionalfonts}
\title{The Computer Scientist as Toolsmith II}
\subtitle{by Fred Brooks}
\author{Antonio, Campos, Chavez, Herrero}
\institute{Ingeniería de Software II}
\date{\today}
\begin{document}

\begin{frame}
\titlepage
\end{frame}

\section{Introduccion}
\begin{frame}{Fred Brooks según Wikipedia}
\begin{center}
He's a software engineer and computer scientist, 
best known for managing the development of IBM's System/360 
family of computers and the OS/360 software support package. 
\end{center}
\end{frame}

\begin{frame}{Fred Brooks según Wikipedia}
\begin{center}
Then later writing candidly about the process in his 
seminal book 
\newline
"The Mythical Man-Month". 
\newline
Brooks has received many awards, 
including the National Medal of Technology in 1985 and the Turing Award in 1999.
\end{center}
\end{frame}
% en 1994 recibió el premio Allen Newell Award cuyo discurso es el que en este paper se publica
\begin{frame}{Toolsmith}
\begin{center}
Persona que hace herramientas
\newline
\newline
Persona que crea programas utilitarios
\end{center}
\end{frame}
\end{document}