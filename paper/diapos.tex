\documentclass{beamer}
\usetheme{default}
\usepackage[utf8]{inputenc}
\usepackage[spanish]{babel}
\usepackage{hyperref}
\usepackage{wrapfig}
\usetheme{Madrid}
\usecolortheme{seahorse}
\usefonttheme{professionalfonts}
\title{The Computer Scientist as Toolsmith II}
\subtitle{by Fred Brooks}
\author{Antonio, Campos, Chavez, Herrero}
\institute{Ingeniería de Software II}
\date{\today}
\begin{document}

\begin{frame}
\titlepage
\end{frame}

\section{Introduccion}
\begin{frame}{Fred Brooks según Wikipedia}
\begin{center}
Es un ingeniero de software y científico de la computación, 
más conocido por dirigir el desarrollo del sistema operativo OS/360 de IBM, 
y después escribir honestamente sobre el proceso en su famoso libro 
\newline
The Mythical Man-Month
\end{center}
\end{frame}
\begin{frame}{Fred Brooks según Wikipedia}
\begin{center}
Recibió muchos premios, 
entre ellos the National Medal of Technology en 1985,
el Premio Turing en 1999,
y el Allen Newell Award en 1994
\end{center}
\end{frame}
% en 1994 recibió el premio Allen Newell Award cuyo discurso es el que en este paper se publica

\begin{frame}{qué significa toolsmith?}
\begin{center}
Persona que hace herramientas
\end{center}
\end{frame}

\begin{frame}{Dos artículos}
\begin{center}
Toolsmith I (1977)y Toolsmith II(1994)
\end{center}
\end{frame}

\begin{frame}{Qué es una ciencia?}
\begin{center}
Científicos construyen para estudiar, 
ingenieros estudian para construir
\end{center}
\end{frame}

\begin{frame}{Qué es nuestra disciplina?}
\begin{center}
Ciencias de la computación 
no es una disciplina científica, 
es una ingeniería
\newline
Relacionada con hacer cosas
\end{center}
\end{frame}

\begin{frame}{Cómo puede un nombre despistarnos?}
\begin{center}
1. Tendemos a pensar que una ciencia tiene mayor 'valor'
que una ingeniería...
\newline
\end{center}
\end{frame}
\begin{frame}{Cómo puede un nombre despistarnos?}
\begin{center}
2. Un nuevo hecho, una nueva ley es un logro que
merece publicación.
\newline
Si reconocemos nuestros artefactos como herramientas, 
probamos su uso y costo, no por su novedad.
\end{center}
\end{frame}
\begin{frame}{Cómo puede un nombre despistarnos?}
\begin{center}
3. Tendemos a olvidar a nuestros usuarios y sus problemas reales,
abstraemos demasiado dejando detrás la esencia del problema real.
\end{center}
\end{frame}

\begin{frame}{La parte Computer está bien}
\begin{center}
La diferencia clave entre ciencas de la computación
y el resto de las ciencias es que somos especialistas en problemas 
que se caracterizan por ser de complejidad arbitraria
\end{center}
\end{frame}

\begin{frame}{La parte Computer está bien}
\begin{center}
Matemáticos se escandalizan por la complejidad, 
mientras físicos o biólogos se escandalizan por 
la arbitrariedad
\end{center}
\end{frame}

\begin{frame}{El regalo de la subcreación}
\begin{center}
El poder de hacer cosas, en imitación a nuestro creador, es un regalo.
\newline
La capacidad y la necesidad de crear nos son dados pra enriquecer nuestras vidas 
y permitirnos enriqueser las de otros.
\end{center}
\end{frame}

\begin{frame}{Sana evolución de la Inteligencia Artificial}
\begin{center}
Haremos máquinas que piensan; haremos Mentes Gigantes.
\end{center}
\end{frame}
\begin{frame}{Sana evolución de la Inteligencia Artificial}
\begin{center}
Se ha logrado sorprendentemente poco por el tiempo y la inversión realizada.
\end{center}
\end{frame}
\begin{frame}{Sana evolución de la Inteligencia Artificial}
\begin{center}
Estos años de experiencia dieron a los trabajadores en AI 
un respeto profundo por el poder de la mente humana
\end{center}
\end{frame}
\begin{frame}{Sana evolución de la Inteligencia Artificial}
\begin{center}
Es momento de reconocer que los objetivos originales de AI 
no fueron extremadamente difíciles,
fueron objetivos que, aunque glamorosos y motivantes, 
enviaron a la disciplina en una dirección equivocada
\end{center}
\end{frame}

\end{document}